% -*- coding: utf-8; -*-

\chapter{Introdução}

A cada semestre, um aluno que está fazendo graduação no Departamento de Informática da PUC-Rio precisa fazer a sua matrícula em disciplinas oferecidas pela universidade para os próximos seis meses. Apesar de algumas restrições como pré-requisitos de disciplinas, o aluno tem ampla liberdade de escolher as disciplinas que melhor se encaixam na sua grade horária. Essa liberdade é uma vantagem devido à disponibilização de uma grade horária flexível para o aluno, porém precisa de um maior esforço de pesquisa e organização deste aluno para que a sua matrícula no próximo período seja feita de forma eficaz segundo critérios do próprio aluno. 

A motivação desse projeto é o estudo, planejamento e desenvolvimento de um sistema de recomendação de disciplinas para o próximo período que auxilie o aluno em sua matrícula ao sugerir interativamente disciplinas com base em dados fornecidos pela universidade e por avaliações informadas por alunos.