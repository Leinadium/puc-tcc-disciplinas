% -*- coding: utf-8; -*-
\chapter{Objetivo}
\label{cha:Objetivo}


O sistema desenvolvido nesse trabalho consiste em uma interface de planejamento de matrícula semelhante ao simulador de matrícula integrado com um algoritmo de recomendação. O sistema disponibiliza as disciplinas do próximo período conforme os dados mais atuais do microhorario. Para que haja uma personalização na recomendação das disciplinas, o sistema permite que o aluno carregue o histórico escolar na universidade afim de fornecer o histórico das disciplinas e seus graus para o algoritmo de recomendação.

O algoritmo de recomendação do sistema recebe as disciplinas oferecidas no próximo período e seus pré-requisitos, o modelo de grade recomendada pelo departamento e o histórico do aluno, e então responde com disciplinas selecionadas. O algoritmo também pode ser personalizado com informações de preferência do usuário.

O sistema armazena o histórico fornecido pelo aluno sem seus dados pessoais, afim de montar uma base de dados contendo exemplos de grades de alunos e suas notas, para que o algoritmo a utilize para obter resultados mais satisfatórios para o aluno usuário do sistema.

O sistema e o algoritmo foi desenvolvido especificamente para alunos do departamento de informática da PUC-Rio devido a dificuldade de abranger todas os currículos nos diferentes departamentos da universidade. Essa restrição também permitiu restrigir o escopo do projeto afim de tentar obter melhores resultados.

% Equation example 1:

% \begin{equation}
% \begin{split}
% \min_u \int_{x_i\in X}\int_{x_j\in X} q_{ij} u_i u_j da da + \int_{x_i\in X}||x' - x_i|| u_i da \\
% s.t. \ \ \ u\in[0,1] \ \ \land  \ \ \int_{x_i\in X}u da = a_0,
% \end{split}
% \end{equation}

% Equation exmaple 2:

% \begin{equation}
% \begin{split}
% \min_{\mathbf{u}} \alpha \mathbf{u}^T \mathbf{A}^T \mathbf{Q} \mathbf{A} \mathbf{u} +  \beta \mathbf{d}^T a' \mathbf{A} \mathbf{u} + \gamma \mathbf{u}^T \mathbf{G}^T \mathbf{G} \mathbf{u} + \delta\mathbf{f}^T a' \mathbf{A} \mathbf{u} \\
% s.t. \ \ \ \mathbf{0} \leq \mathbf{u} \leq \mathbf{1} \land \mathbf{a}^T\mathbf{u}=a_0.
% \end{split}
% \end{equation}

% Equation example 3:
% \begin{align}
% \mathbf{G}=(g_{ij}) = \left\lbrace
% \begin{array}{ll}
% \sum_{f_k\in N_f(f_i)} l_{ik} & i=j\\
% -l_{ij} & e_{ij}\in E\\
% 0 & \text{otherwise}
% \end{array}
% \right.
% \end{align}

% \lstinputlisting[label=mean,title={Mean Filter},caption={Mean Filter},language=R]{codes/mean.R}

% %% Poruguese algorithm
% %\begin{algorithm}
% %\DontPrintSemicolon
% %\Entrada{Malha e quantidade de pontos a ser amostrado}
% %\Saida{Pontos amostrados na malha}
% %\BlankLine
% %\emph{Crie um vetor de números randômicos entre $[0,1]$ com a %quantidade de pontos a ser amostrada e ordene-o}\;
% %\emph{Calcule a área total dos triângulos da malha}\;
% %\For{$i=0$ \KwTo numeroDePontos} {
% %  \emph{Navegue entre as faces acumulando a sua $\frac{area}{areaTotal}$ até achar a face com valor acumulado $\geqslant$ numerosRandomicos[i]}\;
% %  \emph{Pegue um ponto randômico dentro da face utilizando o %método de Turk e adicione no vetor do resultado}\;
% %}
% %\caption{Escolha das amostras inicias}\label{alg:sampling}
% %\end{algorithm}\DecMargin{1em}

% %% enlgish algorithm
% \begin{algorithm}
% \DontPrintSemicolon
% \KwIn{Malha e quantidade de pontos a ser amostrado}
% \KwOut{Pontos amostrados na malha}
% \BlankLine
% \emph{Crie um vetor de números randômicos entre $[0,1]$ com a quantidade de pontos a ser amostrada e ordene-o}\;
% \emph{Calcule a área total dos triângulos da malha}\;
% \For{$i=0$ \KwTo numeroDePontos} {
%   \emph{Navegue entre as faces acumulando a sua $\frac{area}{areaTotal}$ até achar a face com valor acumulado $\geqslant$ numerosRandomicos[i]}\;
%   \emph{Pegue um ponto randômico dentro da face utilizando o método de Turk e adicione no vetor do resultado}\;
% }
% \caption{Escolha das amostras inicias}\label{alg:sampling}
% \end{algorithm}\DecMargin{1em}
