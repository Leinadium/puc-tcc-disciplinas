\chapter{Construção}
\label{cha:Construção}

\section{Tecnologias utilizadas}

Para o armazenamento e consulta dos dados, foi utilizado o banco de dados relacional PostgreSQL \cite{site-postgresql}. O PostgreSQL foi escolhido por ser robusto e eficiente, mas fácil e rápido de configurar. 

O sistema foi separado em interface e API (\textit{Application programming interface}, ou interface de programação da aplicação). Houve a possibilidade de criar o sistema utilizando uma tecnologia que integra a interface e a API, como o Django \cite{site-django}, para evitar a criação de dois serviços separados. Mas como a interface precisava ser bem complexa e interativa, a opção de \textit{templating} do Django não ia ser suficiente. Por isso, a interface foi desenvolvida em Svelte \cite{site-svelte}, um framework em JavaScript \cite{site-js} baseado em componentes, e a API foi desenvolvida com Gin \cite{site-gin}, um framework em Go. \cite{site-go}. A lógica do algoritmo foi implementada dentro do serviço da API.

A coleta dos dados provenientes do Microhorario e a autenticação com o sistema acadêmico universitário foi desenvolvida em serviços menores, cada um com uma única funcionalidade, separados da API (microserviços). Ambos foram desenvolvidos utilizando o framework Flask \cite{site-flask}, utilizando a linguagem \cite{site-python}.

A figura \ref{fig:diagrama-arquitetura} exibe um diagrama da arquitetura do sistema.
% Cada um dos componentes serão explicados nas próximas secções do capítulo.

\begin{figure}[ht]
    \begin{center}
    \includegraphics[width=300pt]{figuras/diagrama-arquitetura.png}
    \caption{Diagrama da arquitetura do sistema}
    \label{fig:diagrama-arquitetura}
    \end{center}
\end{figure}

\section{Coleta de dados}

O o modelo lógico representado na figura \ref{fig:modelo-logico} define a construção de catorze tabelas no banco de dados. O trecho de código \ref{cod:sql-create} contém o modelo físico do banco de dados, ou seja o código SQL que define a modelagem das tabelas.

\lstinputlisting[label=cod:sql-create,title={create.sql},caption={Modelo físico},language=SQL]{codigo/10-create.sql}

Além dos dados gerados pela interação do usuário, o sistema precisa de dados provenientes da faculdade, como as disciplinas, turmas, professores e currículos. Esses dados são coletados ocasionalmente, exceto os currículos, que foram coletados e inseridos somente uma vez.

Como o algoritmo e o sistema foi planejado para o departamento de informática, a quantidade de currículos a ser inserida é pequena. No total, foram inseridos seis currículos: três de engenharia de computação (Currículos 2023, 2018.0 e 2018.1) e três de ciência de computação, disponíveis nas respectivas páginas
\footnote{\url{https://www.puc-rio.br/ensinopesq/ccg/eng_computacao.html}}
\footnote{\url{https://www.puc-rio.br/ensinopesq/ccg/ciencia_computacao.html}}
dos cursos.

Os dados do currículo foram transformados manualmente em um arquivo no formato SQL, executado ao criar o banco de dados. O trecho de código \ref{cod:sql-curriculo} contém um exemplo de um dos seis arquivos criados, um para cada currículo.

\lstinputlisting[label=cod:sql-curriculo,title={curriculo-cie-18-0.sql},caption={Exemplo do arquivo de carga de currículos},language=SQL]{codigo/10-curriculos.sql}

Os dados provenientes das disciplinas e turmas são coletados diretamente através do microhorário. Existe uma biblioteca  
\footnote{Dispon\'ivel em: \url{https://pypi.org/project/microhorario-dl/}}
Python que permite baixar todos as informações disponíveis no microhorário, e opcionalmente coletar as ementas e pré-requisitos que estão disponíveis em página. A biblioteca baixa os dados das disciplinas e turmas em menos de três segundos, mas para também baixar as ementas e pré-requisitos que estão em um outro serviço, a biblioteca demora cerca de 30 minutos. Portanto, foi criado um módulo, chamado \verb|microhorario|, que permite atualizar o banco com as informações básicas das disciplinas e turmas, mas que pode ser alterado para atualizar também as ementas e pré-requisitos. Esse módulo foi configurado de forma a ser executado na sua forma mais simples uma vez a cada hora, e executado na forma completa uma vez por semana.

\section{Definição e implementação do algoritmo}

O algoritmo busca satisfazer as necessidades apresentadas na tabela \ref{tab:entrevista-criterios-peso}. As entrevistas indicaram a importância de cinco categorias (Conteúdo, Professor, Avaliação, Horário e Opinião). Por isso, o algoritmo foi dividido nessas cinco categorias.

O algoritmo se baseia num valor $V[d,u]$ atribuído a cada disciplina, que indica a relevância da disciplina $d$ para o usuário $u$. O valor $V[d,u] \in [0, 1]$,
sendo 1 o maior grau de relevância da disciplina, e 0 o menor. O cálculo de $V$ é uma média ponderada, conforme a equação \ref{equ:algoritmo-valor}.

\begin{multline}
\label{equ:algoritmo-valor}
    V[d,u] = \\
        P_cV_c[d,u] + P_pV_p[d] + P_oV_o[d] + P_hV_h[d,u] + P_aV_a[d]
\end{multline}

Em que $d$ é uma disciplina, $u$ é um usuário, $P_x$ é o valor de uma das cinco categorias para a disciplina $d$ e o usuário $u$, e $P_x$ é o peso de uma das cinco categorias.

%% valor V_c

O valor $V_c[d,u]$ indica o quão relevante é a disciplina para o usuário de acordo com o seu conteúdo. Para isso, o valor é calculado de acordo com o histórico de outros alunos e com o currículo do aluno, conforme a equação \ref{equ:algoritmo-conteudo}.

\begin{equation}
    V_c[d,u] = 
    \begin{dcases}
        \hfil 1.0 & \text{caso} \: d \in Historico[u] \\ 
        \frac{|\, A_{cursou}[d,u] \,|}{|\,  A_{curriculo}[u] \,|}   & \text{caso} \: A_{curriculo}[u] > 0 \\
        0 & \text{caso contrário}
    \end{dcases} \\[20pt]
    \label{equ:algoritmo-conteudo}
\end{equation}

Em que:

\begin{align*}
    & A_{curriculo}[u] = \forall a \in Alunos \,|\, Curriculo[a] = Curriculo[u] \\
    & A_{cursou}[d,u] = \forall a \in A_{curriculo}[u] \,|\, d \in Cursadas[a] \\
\end{align*}

Em que $d$ é uma disciplina, $u$ é um usuário, $Alunos$ é o conjunto de todos os usuários cadastrados no sistema, $Curriculo[u]$ é o currículo do usuário $u$, e $Cursadas[u]$ é o conjunto de disciplinas que o aluno $u$ cursou. Em resumo, o valor $V_c[d,u]$ é o valor máximo caso a disciplina deve ser cursada pelo usuário, ou sendo uma eletiva é a proporção de alunos do mesmo currículo que fizeram esta disciplina. Essa proporção indica se o conteúdo é relevante para alunos semelhantes ao usuário.

%% Valor V_p

O valor $V_p[d]$ indica o quão relevante é a disciplina para o usuário de acordo com o professor. Para isso, o valor é calculado de acordo com as avaliações dos professores das turmas das disciplinas, conforme a equação \ref{equ:algoritmo-professor}.

\begin{equation}
    V_p[d] = \frac{\sum_{p \in P[d]} \displaystyle  \frac{\sum_{a \in Avs[p]} a}{| Avs[p] |}}{| P[d] |} \,/\, 5
    \label{equ:algoritmo-professor}
\end{equation}

Em que $d$ é uma disciplina, $P[d]$ é o conjunto dos professores que lecionam a disciplina $d$, e $Avs[p]$ representa o conjunto das avaliações dos usuários do professor $p$. Em resumo, o valor $V_p[d,u]$ é a média das avaliações de todos os professores que estão lecionando a disciplina, com o valor entre $0$ e $1$.

%% Valor V_o

O valor $V_o[d]$ indica o quão relevante é a disciplina para o usuário de acordo com a opinião dos alunos. A equação é semelhante ao cálculo apresentado na equação \ref{equ:algoritmo-professor}. Porém, nesse caso, é levado em conta a média das avaliações das próprias disciplinas, conforme a equação \ref{equ:algoritmo-opiniao}.

\begin{equation}
    V_p[d] = \frac{\sum_{a \in Op[d]} a} {| Op[d] |} \,/\, 5
    \label{equ:algoritmo-opiniao}
\end{equation}

Em que $d$ é uma disciplina, e $Op[d]$ é o conjunto das avaliações dos usuários da disciplina $d$.

%% Valor V_h

O valor $V_h[d,u]$ indica o quão relevante é a disciplina para o aluno de acordo com o horário. Para isso, o valor é calculado de acordo com os horários ocupados na grade do usuário e os horários das disciplinas, conforme a equação \ref{equ:algoritmo-horario}

\begin{equation}
    V_h[d,u] = 
    \begin{dcases}
        \frac{|\, T_{possiveis}[d,u] \,|}{|\,  T[d] \,|}   & \text{caso} \: T[d] > 0 \\
        \hfil 0 & \text{caso contrário}
    \end{dcases} \\[20pt]
    \label{equ:algoritmo-horario}
\end{equation}

Em que $d$ é uma disciplina, $u$ é um usuário, $T_{possiveis}[d,u]$ é o conjunto de turmas da disciplina $d$ que se encaixam na grade do usuário $u$, e $T[d]$ é a quantidade de turmas da disciplina $d$. Em resumo, $V_h[d, u]$ é a porcentagem de turmas da disciplina que se encaixam na grade do usuário.

Por último, o valor $V_a[d]$ indica o quão relevante é a disciplina para o aluno de acordo com os graus (a nota final, considerando as provas do aluno durante o semestre) da disciplina. O cálculo é semelhante às equações \ref{equ:algoritmo-professor} e \ref{equ:algoritmo-opiniao}, conforme a equação \ref{equ:algoritmo-avaliacao}.

\begin{equation}
    V_p[d] = \frac{\sum_{g \in Graus[d]} g} {| Graus[d] |} \,/\, 100 
    \label{equ:algoritmo-avaliacao}
\end{equation}

Em que $d$ é uma disciplina e $Graus[d]$ é o conjunto das notas da disciplina $d$.

Por fim, os pesos $P_x$ da média ponderada foram baseados na tabela \ref{tab:entrevista-criterios-peso}. Cada peso é a soma dos pesos dos critérios, de acordo com as equações a seguir.

\begin{align}
    & P_c = \frac{41}{41 + 19 + 8 + 28 + 21} \\[10pt]
    & P_p = \frac{28}{41 + 19 + 8 + 28 + 21} \\[10pt]
    & P_o = \frac{ 8}{41 + 19 + 8 + 28 + 21} \\[10pt]
    & P_h = \frac{19}{41 + 19 + 8 + 28 + 21} \\[10pt]
    & P_a = \frac{21}{41 + 19 + 8 + 28 + 21}
\end{align}

%% TODO: outras categorias

\section{Implementação da API}

\section{Implementação da interface}