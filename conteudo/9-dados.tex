\chapter{Dados do sistema}
\label{cha:Dados do sistema}

Para satisfazer os requisitos e os casos de uso, foi necessário modelar os dados disponíveis ao sistema e ao algoritmo.

\section{Diagrama de entidades-relacionamento}

A imagem \ref{fig:diagrama-classes} exibe o diagrama de entidades-relacionamento (ER), que descreve os dados no modelos de entidades (coisas de eresse) e seus relacionamentos. Este modelo segue a notação do Peter Chen \cite{peter-chen}. 

\begin{figure}[ht]
    \begin{center}
    \includegraphics[width=390pt]{figuras/diagrama-er-chen.png}
    \caption{Diagrama do modelo entidade-relacionamento}
    \label{fig:diagrama-classes}
    \end{center}
\end{figure}

\section{Modelo lógico}

A imagem \ref{fig:modelo-logico} exibe o diagrama lógico dos dados com base no modelo Entidade-Relacionamento. Ele representa a estrutura implementada no banco de dados, com suas tabelas e chaves primárias (PK) e chaves estrangeiras (FK).

\begin{figure}[ht]
    \begin{center}
    \includegraphics[width=390pt]{figuras/diagrama-logico.png}
    \caption{Diagrama do modelo lógico dos dados}
    \label{fig:modelo-logico}
    \end{center}
\end{figure}

\section{Dicionário de dados}

O dicionário de dados possui a função de categorizar os dados. 
Ele é uma coleção de metadados do modelo físico. 
O dicionario de dados do sistema está disponível na tabela \ref{tab:dicionario-dados}. 
Nele está representado todas as colunas observadas no diagrama lógico da figura \ref{fig:modelo-logico}.
O tipo do dado é unico para o nome da coluna, o que significa que o mesmo nome da coluna em duas tabelas diferentes representa o mesmo tipo de dado. 
Por exemplo, o tipo de dado na coluna \textit{cod-usuario} é o mesmo na tabela \textit{grades}, \textit{usuarios}, \textit{avaliacoes-disciplina} e \textit{avaliacoes-professor}.

\begin{longtable}{ | >{\raggedright}m{0.22\columnwidth} | >{\raggedright}m{0.38\columnwidth} | >{\raggedright}m{0.1\columnwidth} | >{\raggedright}m{0.2\columnwidth} | }
    
    \hline\textbf{Coluna} & \textbf{Descrição} & \textbf{Tipo} & \textbf{Domínio}\tabularnewline\hline\hline
    \endhead
    
    % \multicolumn{5}{|c|}{Dicionario de dados}\tabularnewline\hline\hline\endfirsthead
    % \hline\hline
    % \multicolumn{2}{|c|}{Dicionario de dados (continuação)}\tabularnewline\hline\hline\endhead
    % \hline\endfoot
    \hline\caption{Dicionário de dados}\endlastfoot

    cod\textunderscore curriculo & Identificador do currículo. Exemplo: "CEGBCO20180" & string & Sem restrição.\tabularnewline\hline
    cod\textunderscore depto & Abreviação de três letras do departamento, conforme disponibilizado no microhorario. Exemplo: "ENG". & string & Três letras maiúsculas.\tabularnewline\hline
    cod\textunderscore disciplina & Código da disciplina. Exemplo: "INF1011". & string & Três letras maiúsculas (não necessariamente um departamento) seguidas de 4 números.\tabularnewline\hline
    cod\textunderscore disc\textunderscore depen & Códigos da disciplinas que fazem parte de um grupo de pré-requisitos de outra disciplina. & Array de strings & Mesmo do cod-disciplina.\tabularnewline\hline
    cod\textunderscore disc\textunderscore orig & Código da disciplina que possui algum pré-requisito. & string & Mesmo do cod-disciplina.\tabularnewline\hline
    cod\textunderscore grade & Código da grade, gerado ao salvar uma nova grade. & string & 8 caracteres em formato base64.\tabularnewline\hline
    cod\textunderscore turma & Código da turma. & string & 3 caracteres.\tabularnewline\hline
    cod\textunderscore usuario & Código do usuário, representado por sua matrícula & string & 7 números.\tabularnewline\hline
    conteudo & Conteúdo codificado da grade horária, que é decodificado pela interface ao exibir. & string & Sem restrição.\tabularnewline\hline
    creditos & Quantidade de créditos da disciplina. & int & Valor entre 0 a 30.\tabularnewline\hline
    data\textunderscore avaliacao & Data em que a avaliação foi efetuada pelo aluno. & date & Sem restrição. Pode ser nulo.\tabularnewline\hline
    dia & Dia da semana que a turma é oferecida & string & "SEG", "TER", "QUA", "QUI", "SEX", "SAB" ou "xxx" (sem dia definido).\tabularnewline\hline
    destino & Destino das vagas disponíveis. Exemplo: "QQC" (Qualquer curso), "BCO" (Bacharelado em Engenharia de Computação), entre outros. & string & Três letras maiúsculas. Pode ser nulo.\tabularnewline\hline
    ementa & Ementa da disciplina. & string & Sem restrição. Pode ser nulo.\tabularnewline\hline
    grau & Grau do aluno na disciplina. & int & Entre 0 a 100. Pode ser nulo.\tabularnewline\hline
    grupo\textunderscore prereq & Um identificador do grupo do pré-requisito. Um aluno só pode cursar a disciplina se todas as disciplinas do mesmo grupo foram cursadas. & int & Sem restrição.\tabularnewline\hline
    hora\textunderscore fim & Hora do fim da aula da turma & int & Valor entre 9 e 23, deve ser maior que hora-inicio. Pode ser nulo (sem hora de fim).\tabularnewline\hline
    hora\textunderscore inicio & Hora do inicio da aula da turma & int & Valor entre 7 e 21, ou 0 (sem hora de início).\tabularnewline\hline
    nome\textunderscore curriculo & Nome curto do currículo. Exemplo: "Engenharia | 19.0". & string & Sem restrição.\tabularnewline\hline
    nome\textunderscore depto & Nome do departamento, conforme disponibilizado no microhorario. Exemplo: "Engenharia". & string & Sem restrição.\tabularnewline\hline
    nome\textunderscore disciplina & Nome da disciplina. Exemplo: "Semântica de Linguagens". & string & Sem restrição.\tabularnewline\hline
    nome\textunderscore professor & Nome do professor, conforme disponibilizado no microhorário & string & Sem restrição.\tabularnewline\hline
    nome\textunderscore usuario & Nome do usuário & string & Sem restrição.\tabularnewline\hline
    nota\textunderscore avaliacao & Nota da disciplina ou professor avaliado. & int & Valor entre 1 e 5.\tabularnewline\hline
    semestre & Semestre recomendado para cursar a disciplina de acordo com o currículo. & int & Valor entre 1 e 10. Pode ser nulo.\tabularnewline\hline
    shf & Quantidade de horas "Sem Horário Fixo" de uma disciplina. & int & Valor maior ou igual a zero. Pode ser nulo.\tabularnewline\hline
    vagas & Quantidade de vagas disponíveis. & int & Valor maior que 0.
    \label{tab:dicionario-dados}
\end{longtable}
