% -*- coding: utf-8; -*-

\chapter{Requisitos}
\label{cha:Requisitos}

Os requisitos estão divididos em requisitos relacionados ao sistema e requisitos relacionados ao algoritmo.
Os requisitos do sistema dizem respeito ao sistema que disponibiliza o serviço de montagem de grade horária do próximo período, assim como um serviço de avaliar disciplinas e professores, que existe para satisfazer as necessidades observadas nos capítulos anteriores. Os requisitos do algoritmo dizem respeito às entradas e saídas do algoritmo afim de se produzir uma recomendação adequada de acordo as necessidades observadas no capítulo \ref{cha:validacao-problema}.

\section{Do algoritmo}

Os requisitos do algoritmo de recomendação de disciplinas precisam satisfazer os critérios de escolha de disciplinas mais relevantes, discutidos no capítulo \ref{sec:escolha-disciplinas}, ou seja, recomendar de acordo com o conteúdo da disciplina, horários disponíveis, opiniões de amigos, o professor, e por último o método de avaliação.
Além disso, para satisfazer a preferência de disciplinas eletivas discutidas no capítulo \ref{sec:disciplinas-eletivas}, os requisitos precisam satisfazer a preferência de recomendar disciplinas com base na sua facilidade.

A lista de requisitos do algoritmo está disponível na tabela \ref{tab:req-algoritmo}.

\begin{table}[!ht]
    \begin{center}
        \begin{tabular}{ | m{0.1\textwidth} | p{0.8\textwidth} | }  
            \hline
            \textbf{RF01} & O algoritmo deve receber as disciplinas e turmas oferecidas no próximo semestre conforme o microhorário.\tabularnewline\hline
            \textbf{RF02} & O algoritmo deve receber opcionalmente a grade curricular do curso que o aluno está cursando.\tabularnewline\hline
            \textbf{RF03} & O algoritmo deve receber as avaliações das disciplinas e professores fornecidas pelos usuários do sistema.\tabularnewline\hline
            \textbf{RF04} & O algoritmo deve receber opcionalmente o histórico escolar do aluno.\tabularnewline\hline
            \textbf{RF05} & O algoritmo deve receber disciplinas já selecionadas pelo aluno.\tabularnewline\hline
            \textbf{RF06} & O algoritmo deve retornar recomendações de disciplinas com base nas entradas fornecidas.\tabularnewline\hline
            
            \textbf{RNF1} & O algoritmo deve retornar as recomendações em menos de 3 segundos.\tabularnewline\hline
            \textbf{RNF2} & O algoritmo deve ser determinístico, ou seja, retorna as mesmas recomendações para as mesmas entradas.\tabularnewline\hline
        \end{tabular}
    \end{center}
    \caption{Requisitos do algoritmo}
    
    \label{tab:req-algoritmo}
\end{table}

\section{Do sistema}

O sistema precisa satisfazer as necessidades do usuário e também as necessidades do algoritmo, pois o sistema hospeda o algoritmo de recomendação.
Os requisitos do sistema estão disponíveis na tabela \ref{tab:req-sistema}.

\begin{table}[!ht]
    \begin{center}
        \begin{tabular}{ | m{0.1\textwidth} | p{0.8\textwidth} | }  
            \hline
            \textbf{RF07} & O sistema deve permitir que o usuário crie uma grade horária para o próximo período.\tabularnewline\hline
            \textbf{RF08} & O sistema deve permitir que o usuário submita seu histórico escolar.\tabularnewline\hline
            \textbf{RF09} & O sistema deve permitir que o usuário selecione turmas das disciplinas para compor sua grade horária.\tabularnewline\hline
            \textbf{RF10} & O sistema deve permitir que o usuário autenticado armazene a sua grade horária finalizada.\tabularnewline\hline
            \textbf{RF11} & O sistema deve permitir que o usuário compartilhe a sua grade horária finalizada. \tabularnewline\hline
            \textbf{RF12} & O sistema deve permitir que o usuário recupere uma grade horária montada a partir de um link compartilhado.\tabularnewline\hline
            
            \textbf{RF13} & O sistema deve permitir que o usuário inicie uma sessão autenticada via o Sistema Acadêmico Universitário (SAU).\tabularnewline\hline
            \textbf{RF14} & O sistema deve permitir que o usuário finalize uma sessão autenticada.\tabularnewline\hline
            \textbf{RF15} & O sistema deve permitir que o usuário autenticado avalie uma disciplina.\tabularnewline\hline
            \textbf{RF16} & O sistema deve permitir que o usuário autenticado avalie um professor de uma disciplina.\tabularnewline\hline
            \textbf{RF17} & O sistema deve permitir que o usuário autenticado altere uma avaliação feita anteriormente.\tabularnewline\hline

            \textbf{RF18} & O sistema deve periodiacamente sincronizar as vagas e turmas com o microhorário.\tabularnewline\hline
            \textbf{RF19} & O sistema deve periodiacamente adicionar novas disciplinas conforme disponibilizado no microhorário.\tabularnewline\hline

            %% \textbf{RNF3} & O sistema deve estar disponível publicamente aos alunos da universidade.\tabularnewline\hline
            \textbf{RNF3} & O sistema deve suportar pelo menos 40 usuários simultâneos.\tabularnewline\hline
        
        \end{tabular}
    \end{center}
    \caption{Requisitos do sistema}
    
    \label{tab:req-sistema}
\end{table}
