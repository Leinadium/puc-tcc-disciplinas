\chapter{Conclusão}
\label{cha:Conclusão}

A partir da construção e dos testes com os usuários, entendeu-se que objetivo do projeto de auxiliar alunos em seus períodos de matrícula foi alcançado.
Ter um sistema para planejar sua matrícula com todas as informações necessárias organizadas em um só lugar, em vez de em três ou quatro serviços diferentes, é inovador
para esta universidade. O algoritmo de recomendação cumpre o seu papel em recomendar disciplinas se utilizando de diversas fontes de informação, facilitando
o planejamento do usuário aluno da universidade.

Durante todo o processo de entrevistas, modelagem e construção, tanto o sistema como o algoritmo passaram por diversas mudanças. As entrevistas revelaram que os
processos de preferência de cada aluno são diferentes, e que mesmo tratando-se de um só departamento, cada aluno tem sua preferência pessoal, de forma que nenhum
sistema ou algoritmo pode satisfazer todos os usuários de forma completa e eficaz.

\section{Estudos futuros}

O projeto do sistema possui uma gama de possíveis estudos futuros. Em um primeiro momento, deve-se implementar as funcionalidades descobertas durante o processo de desenvolvimento, como grupo de disciplinas optativas e co-requisitos, e as sugestões complexas provenientes dos usuários na fase de testes. Além disso, foram planejados algumas temas a serem estudados, conforme a lista a seguir:

%\label

\begin{enumerate}
    \item \textbf{Disciplinas optativas e co-requisitos}: Conforme explicado nas seções \ref{sec:Disciplinas optativas} e \ref{sec:Co-requisitos}, o projeto não considera disciplinas optativas nem co-requisitos de disciplinas. Uma reformulação da arquitetura dos dados é uma oportunidade de estudo para adequar essas características ao sistema.

    \item \textbf{Inteligência Artificial}: O algoritmo foi construído através de um modelo matemático estático, com pesos fixos. Um estudo focado em inteligência artificial para modelar um novo algoritmo ou uma nova escolha de pesos pode otimizar as recomendações.
    
    \item \textbf{Performance da API}: Mesmo satisfazendo os requisitos planejados, a performance da API pode ser melhor otimizada. Melhores consultas, utilização de cache, e visões materializadas são possíveis campos de estudo importantes em um tempo futuro.
    
    \item \textbf{Geração automática da grade}: Após validar com um maior grupo de usuários e verificar a robustez das recomendações, poderia haver uma funcionalidade de escolher automaticamente a melhor recomendação de forma automática, gerando uma grade somente com as disciplinas recomendadas.
    
    \item \textbf{Estudar outros pesos}: Apesar das entrevistas revelarem cinco categorias de uma escolha de uma disciplina, essas podem não ser as melhores para modelar uma recomendação. Um estudo futuro pretende efetuar mais estrevistas, coletar mais informações e possivelmente adicionar ou remover pesos do algoritmo, de forma a otimizar a recomendação das disciplinas.
\end{enumerate}

\section{Código completo}

% \footnote{Dispon\'ivel em: \url{https://pypi.org/project/microhorario-dl/}}

Todo os códigos desenvolvidos para o projeto estão disponíveis publicamente em repositórios do Github:

\begin{itemize}
    \item \verb|github.com/Leinadium/puc-disciplinas|: Código do sistema e do algoritmo e suas respectivas documentações;
    \item \verb|github.com/Leinadium/microhorario-dl|: Código da biblioteca para interação com o microhorario;
    \item \verb|github.com/Leinadium/puc-tcc-disciplinas|: Código para a criação e documentação do relatório final.
\end{itemize}